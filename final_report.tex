% Options for packages loaded elsewhere
\PassOptionsToPackage{unicode}{hyperref}
\PassOptionsToPackage{hyphens}{url}
%
\documentclass[
]{article}
\title{Exploring the seasonal variation in electric vehicle charging in
New Zealand}
\author{Pablo Paulsen}
\date{18/02/2022}

\usepackage{amsmath,amssymb}
\usepackage{lmodern}
\usepackage{iftex}
\ifPDFTeX
  \usepackage[T1]{fontenc}
  \usepackage[utf8]{inputenc}
  \usepackage{textcomp} % provide euro and other symbols
\else % if luatex or xetex
  \usepackage{unicode-math}
  \defaultfontfeatures{Scale=MatchLowercase}
  \defaultfontfeatures[\rmfamily]{Ligatures=TeX,Scale=1}
\fi
% Use upquote if available, for straight quotes in verbatim environments
\IfFileExists{upquote.sty}{\usepackage{upquote}}{}
\IfFileExists{microtype.sty}{% use microtype if available
  \usepackage[]{microtype}
  \UseMicrotypeSet[protrusion]{basicmath} % disable protrusion for tt fonts
}{}
\makeatletter
\@ifundefined{KOMAClassName}{% if non-KOMA class
  \IfFileExists{parskip.sty}{%
    \usepackage{parskip}
  }{% else
    \setlength{\parindent}{0pt}
    \setlength{\parskip}{6pt plus 2pt minus 1pt}}
}{% if KOMA class
  \KOMAoptions{parskip=half}}
\makeatother
\usepackage{xcolor}
\IfFileExists{xurl.sty}{\usepackage{xurl}}{} % add URL line breaks if available
\IfFileExists{bookmark.sty}{\usepackage{bookmark}}{\usepackage{hyperref}}
\hypersetup{
  pdftitle={Exploring the seasonal variation in electric vehicle charging in New Zealand},
  pdfauthor={Pablo Paulsen},
  hidelinks,
  pdfcreator={LaTeX via pandoc}}
\urlstyle{same} % disable monospaced font for URLs
\usepackage[margin=1in]{geometry}
\usepackage{longtable,booktabs,array}
\usepackage{calc} % for calculating minipage widths
% Correct order of tables after \paragraph or \subparagraph
\usepackage{etoolbox}
\makeatletter
\patchcmd\longtable{\par}{\if@noskipsec\mbox{}\fi\par}{}{}
\makeatother
% Allow footnotes in longtable head/foot
\IfFileExists{footnotehyper.sty}{\usepackage{footnotehyper}}{\usepackage{footnote}}
\makesavenoteenv{longtable}
\usepackage{graphicx}
\makeatletter
\def\maxwidth{\ifdim\Gin@nat@width>\linewidth\linewidth\else\Gin@nat@width\fi}
\def\maxheight{\ifdim\Gin@nat@height>\textheight\textheight\else\Gin@nat@height\fi}
\makeatother
% Scale images if necessary, so that they will not overflow the page
% margins by default, and it is still possible to overwrite the defaults
% using explicit options in \includegraphics[width, height, ...]{}
\setkeys{Gin}{width=\maxwidth,height=\maxheight,keepaspectratio}
% Set default figure placement to htbp
\makeatletter
\def\fps@figure{htbp}
\makeatother
\setlength{\emergencystretch}{3em} % prevent overfull lines
\providecommand{\tightlist}{%
  \setlength{\itemsep}{0pt}\setlength{\parskip}{0pt}}
\setcounter{secnumdepth}{-\maxdimen} % remove section numbering
\usepackage{float}
\floatplacement{figure}{H}
\usepackage{comment}
\ifLuaTeX
  \usepackage{selnolig}  % disable illegal ligatures
\fi

\begin{document}
\maketitle

\[ E_{m,R} = \eta_{m,R} \times d_{m,R}\]

\hypertarget{data-exploration}{%
\subsection{Data Exploration}\label{data-exploration}}

\hypertarget{flip-the-fleet-data-exploration}{%
\subsubsection{Flip the Fleet Data
Exploration}\label{flip-the-fleet-data-exploration}}

Distance traveled and vehicle efficiency (km/kWh) by month, as well as
the region of the vehicle was collected from the on-board computers of
1259 vehicles between 2017 and 2021 as part of the `Flip the Fleet'
project.

A monthly weighted average was calculated for the whole of New Zealand
and then for each region of NZ. The monthly averages were weighted using
the distance traveled to give more weighting to vehicles with higher km
traveled in that month. This was done using the formula
\[\bar{x} = \frac{\sum_{i}^{n} (d_i\times x_i)}{\left(\sum_{i}^{n} d_i\right)\times n}\]

Power consumption (Wh/km) was calculated using the efficiency (km/kWh).
This will be used instead of efficiency in the modelling for reasons
that will become apparent later in the analysis.

\begin{figure}
\centering
\includegraphics{final_report_files/figure-latex/consum_plot-1.pdf}
\caption{Time series of EVs weighted mean consumption using Flip the
Fleet data from all NZ regions\label{fig:consum_plot}}
\end{figure}

Figure \ref{fig:consum_plot} shows there is a clear seasonal trend in
the monthly average consumption of Flip the Fleets vehicles from all
regions of NZ.

A time series Decomposition is used to isolate the seasonal trend in
consumption from the overall trend. This can be done for all regions of
NZ combined and also for each region independently.

\begin{figure}
\centering
\includegraphics{final_report_files/figure-latex/consum_decomp_plot-1.pdf}
\caption{Multiplicative time series decomposition of Flip the Fleet
average consumption for all of NZ\label{fig:consum_decomp_plot}}
\end{figure}

\begin{figure}
\centering
\includegraphics{final_report_files/figure-latex/acf_consum-1.pdf}
\caption{Autocorrelation plot of Flip the Fleet average consumption for
all of NZ\label{fig:acf_consum}}
\end{figure}

The time series decomposition (Figure \ref{fig:consum_decomp_plot})
shows a very clear seasonal trend. The autocorrelation plot (Figure
\ref{fig:acf_consum}) shows that this yearly trend is significant. This
seasonal trend goes from 0.96 times the mean consumption in February to
1.07 times the mean consumption in July, a peak to peak difference of
10.7\%.

As NZ weather differs significantly by region, to test the hypothesis
that EV consumption is correlated with heating degree days we must limit
the comparison to a single region of Flip the Fleet data and compare it
to that regions weather at the same period of time.

In order to do this, hourly weather data from 2017 to 2021 was collected
from the NIWA National Climate Database for 14 regions around New
Zealand that best correspond to the regions of the Flip the Fleet
vehicles. The base temperatures were selected to represent the range of
comfortable temperatures for most people, as research shows that a
majority of the seasonal variation in EV efficiency is due to cabin
temperature control\cite{ev_range}. Using the regional hourly
temperatures, monthly heating degree days (HDD) and cooling degree days
(CDD) were imputed using base temperatures of 16\(^\circ\)C and
22\(^\circ\)C respectively. Monthly average temperature were also
calculated.

The HDD and CDD was then divided by the length of the month so that HDD
and CDD corresponds to average heating degrees days per day for the
month. This is so that when comparing to other statistics, such as
efficiency that are averaged out rather than summed, there is less bias.

The calculated monthly weather statistics by region was then added to
the monthly EV data based on the regions of vehicle. This assumes that
most vehicles stay in their own region for a majority of the time.

\begin{figure}
\centering
\includegraphics{final_report_files/figure-latex/consum_HDD_plot-1.pdf}
\caption{Auckland seasonal HDD and EV Consumption
decompostions\label{fig:consum_HDD_plot}}
\end{figure}

Auckland is used as an example to compare correlation between HDD and
consumption as it has the largest amount of data and is of most interest
to Vector. Within Auckland Figure \ref{fig:consum_HDD_plot} shows very
clearly that HDD and consumption of EVs are highly correlated. There is
a slight increase in consumption in January and February and it can be
questioned if that is due to AC usage which would decrease range
\cite{ev_range} or other factors such as holiday travel, often involving
highway driving which EVs are generally less efficient at
\cite{ev_highway}. This effect is not obvious in the overall trend this
could be as Auckland for the most part is a warmer climate than the rest
of NZ.

\begin{figure}
\centering
\includegraphics{final_report_files/figure-latex/temp_consum_plot-1.pdf}
\caption{Auckland monthly average consumption by avg
temperature\label{fig:temp_consum_plot}}
\end{figure}

Further looking into Auckland consumption by weather, Figure
\ref{fig:temp_consum_plot} shows a decreasing consumption up to around a
monthly average temperature of 17.5°C. However, increasing monthly
average temperature past this there appears to be a trend towards
increasing EV consumption. As stated previously, research
\cite{ev_range} suggested AC also increases consumption of the EV. This
suggests it may be worth including cooling degree days and heating
degree days in analysis. This could also be useful to explain the points
well above the trend line that may be from a month where there was both
cold and warm days contributing to a high usage of cabin temperature
control, increasing consumption, but average temperature would not be
able to show this.

\hypertarget{nz-vkt-data-exploration}{%
\subsubsection{NZ VKT Data Exploration}\label{nz-vkt-data-exploration}}

If we know EV consumption has a seasonal trend, in order to see how this
will affect the grid we need to see how this correlates with NZ
populations driving patterns using Vehicles Kilometers Traveled (VKT).

To explore the seasonal trend in fuel usage in NZ, fuel trade data
\cite{fuel_trade} from the Ministry of Business, Innovation and
Employment (MBIE) is used. This data set includes quarterly fuel usage
data broken down by fuel type and sector. This allows the isolation of
petrol usage in domestic land transport, which should give an accurate
representation of the fuel usage by light passenger vehicles. Fuel trade
data from 2020 was excluded as lockdowns were not an accurate
representation of the general driving patterns of the NZ population.

\begin{figure}
\centering
\includegraphics{final_report_files/figure-latex/petrol_ts-1.pdf}
\caption{Multiplicative time series decomposition of petrol usage in
domestic land transport\label{fig:petrol_ts}}
\end{figure}

Figure \ref{fig:petrol_ts} decomposition shows a seasonal trend in
petrol usage, however, it is of relatively small magnitude compared to
the random variations suggesting this trend may not be significant.

\begin{figure}
\centering
\includegraphics{final_report_files/figure-latex/acf_petrol-1.pdf}
\caption{Autocorrelation of petrol usage in domestic land
transport\label{fig:acf_petrol}}
\end{figure}

The autocorrelation plot (figure \ref{fig:acf_petrol}) suggest that
there might be a slight trend in petrol usage however it does not appear
to be of much significance.

Fuel trade data can be compared to the VKT data from NZTA. VKT data
including quarterly data of 10 regions plus one ``other'' region was
given by Haobo Wang from NZTA for use in this project. Further yearly
data for VKT of the ``other'' regions, the vehicle fuel type and vehicle
type was collected from the publicly available fleet statistics page on
NZTA's website. The quarterly VKT data was then multiplied by the
proportion of VKT that was attributed to light passenger vehicles in
that year.

\begin{figure}
\centering
\includegraphics{final_report_files/figure-latex/VKT_ts-1.pdf}
\caption{Decomposition of NZ all regions passenger VKT Time
Series\label{fig:VKT_ts}}
\end{figure}

Figure \ref{fig:VKT_ts} decomposition of the NZ all regions combined VKT
data shows a clear seasonal trend, albeit smaller than the trend from
the fuel sales data. There is, however, clearly a large amount of
smoothing going on with this data. This is shown in a couple of
different ways including:

\begin{itemize}
\item The drop of VKT due to lockdown which started in 2020 March is already visible in the data from early 2019. 
\item Related to the previous point, the Random component of Time Series Decomposition shows only a 10\% decrease in VKT spread out over a 1 year period from lockdown, compared to 30\% drop in fuel usage during only 1 quarter shown in the MIBE fuel trade data. 
\item Random variation in MIBE fuel trade data shows around a 3 times greater random variation. There could be a seasonal effect on fuel efficiency which could change seasonal fuel trend relative to VKT, but there is no reason there would be any significant randomness in fuel efficiency so randomness should be of similar magnitude.
\end{itemize}

This smoothing likely occurs due to the method of data collection using
the odometer readings during WoF/CoF. For a majority of vehicles WoF is
only done once a year and in the case of new cars that could be up to 3
years. This likely causes the data to show less seasonal trend than may
exist in the real world.

Looking at the long term trend, VKT remained largely flat between 2004
and 2012 after which there was a steady but significant increase until
2019. After this, there is a decrease in VKT due to lockdown, which in
this data set for the above reasons likely started showing its effects
in 2019.

\begin{figure}
\centering
\includegraphics{final_report_files/figure-latex/petrol_VKT_vs_eff-1.pdf}
\caption{NZ Seasonal Component Decompostions}
\end{figure}

Looking at the Seasonal trend of Petrol Usage and VKT data from NZTA, we
can see an obvious decrease in the winter months with a peak in the 4th
quarter likely corresponding to holiday travel. Petrol Usage shows this
variation to be much larger than the VKT data from NZTA. It is unclear
whether this would be due to the smoothing effect as was previously
discussed regarding the NZTA data, or perhaps a change in efficiency for
petrol vehicle by seasons similar to that of the EV. Combining these 2
data sets it is reasonable to suggest that in New Zealand, compared to
the winter (Q2 and Q3) VKT, the true VKT in the summer (Q1 and Q4) is
between 1.3\% higher, as suggested by the VKT data from NZTA, to 5\%
higher, according to the petrol usage data.

Looking at the seasonal trend of EV consumption we can see a much larger
increase in consumption in the winter months, with average consumption
in July being 10.7\% higher consumption than in February. From the plot
we can see that when consumption of EVs increases, VKT goes down,
suggesting that some increase in total power usage due to EVs increase
in consumption will be countered by the decrease in VKT. However, the
increase in consumption is much larger than than the decrease in VKT.
This, combined with the fact that winter is when our electricity grid in
New Zealand is already under strain due to heating demand, suggests that
if we ignore the relatively small change in VKT in our model we can
effectively model a worst case scenario. Thus we propose that distance
(\(d_{R}\)) in our model is given by the yearly regional VKT data from
2019 and has no dependency on month.

\hypertarget{model}{%
\subsection{Model}\label{model}}

Based on the findings from the data exploration, we propose a model for
monthly power usage (\(E_{m,R}\)) for each month and region given by the
formula below:

\[ E_{m,R} = \eta_{m,R} \times d_{R}\]

As stated in the data exploration distance (\(d_{R}\)) in our model is
given by the yearly regional VKT data from 2019 and has no dependency on
month.

Based on the data exploration we propose we model EV consumption
(\(\eta_{m,R}\)) using a linear model given by the formula:

\[ \eta_{m,R,C} = \beta_{CDD}{CDD}_{m,R} + \beta_{HDD}{HDD}_{m,R} + R + C + \beta_0 + \epsilon \]

where \({CDD}_{m,R}\) and \({HDD}_{m,R}\) is the number of CDD and HDD
each month in each region, R is a constant given for each region, C is a
constant given for each model of car and \(\beta_0\) is a constant
intercept. This means \emph{expected} consumption can be given by the
formula:

\[ \eta_{m,R,C} = \beta_{CDD}{CDD}_{m,R} + \beta_{HDD}{HDD}_{m,R} + B_{R,C} \]

where \(B_{R,C}\) is effectively a baseline efficiency of a particular
vehicle model is a particular region with no HDD or CDD.

A different intercept is used for each model of car as a majority of the
variation in efficiency will be due to different vehicle models,
therefore, including the vehicle model allows for much better model fit
and smaller confidence intervals. A different intercept is also used for
each weather region as weather might be measured in a cold or hot
section of region and also the region may have more or less hill/highway
which could influence driving patterns impacting efficiency. However the
Gradient of HDD term (\(\beta_{HDD}\)) and CDD term(\(\beta_{CDD}\)) is
kept same for all regions and models as this is the number we are trying
to find to see how the number of HDD and CDD affect the efficiency of
the EV.

As with the case of the adjusted monthly average power consumption
(Wh/km) in the linear model a weighting is added to the points in order
to give more weighting to cars with longer distance traveled. This may
give a slight bias towards EVs with proportionally higher highway
mileage. However, from the electricity grids perspective it makes sense
to give less weighting to cars that have traveled 0 or very low km.
Mathematically this means instead of estimating the coefficients by
minimizing the residual sum of squares (RSS) given by the function
\(\sum_{i =1}^{n}(\eta_{i}-\hat{\eta}_{i})^2\) we minimize the function
\(\sum_{i =1}^{n}d \cdot(\eta_{i}-\hat{\eta}_{i})^2\) where \(d\) is the
distance traveled by a car in that month, \(\eta_{i}\) is the actual
power consumption, and \(\hat{\eta}_{i}\) is the power consumption of
that vehicle as predicted by the model.

EV consumption is modelled with a linear model as with the correct base
temperature the usage of power to warm/cool the cabin should be roughly
linear to the HDD/CDD \cite{HDD_est}. This would allow energy used to
heat/cool the car to be isolated for analysis from drivetrain power
consumption. Conceptually it makes sense that extra power usage due to
heating/cooling demand to be independent from drivetrain demand as
unlike in traditional internal combustion engine (ICE) vehicles where
the energy to heat and cool the cabin comes from the engine, an EVs heat
pump or resistive heater and AC can draw power from the battery
independently of the engine. Unfortunately, this linear correlation may
break down as cars unlike houses or buildings are often only used at
particular hours of the day for short periods so this may break down or
have more dependency towards the temperature at times such as the
morning or evening commute hours.

\begin{longtable}[]{@{}
  >{\raggedright\arraybackslash}p{(\columnwidth - 8\tabcolsep) * \real{0.41}}
  >{\raggedleft\arraybackslash}p{(\columnwidth - 8\tabcolsep) * \real{0.14}}
  >{\raggedleft\arraybackslash}p{(\columnwidth - 8\tabcolsep) * \real{0.16}}
  >{\raggedleft\arraybackslash}p{(\columnwidth - 8\tabcolsep) * \real{0.13}}
  >{\raggedleft\arraybackslash}p{(\columnwidth - 8\tabcolsep) * \real{0.16}}@{}}
\toprule
\begin{minipage}[b]{\linewidth}\raggedright
~
\end{minipage} & \begin{minipage}[b]{\linewidth}\raggedleft
Estimate
\end{minipage} & \begin{minipage}[b]{\linewidth}\raggedleft
Std. Error
\end{minipage} & \begin{minipage}[b]{\linewidth}\raggedleft
t value
\end{minipage} & \begin{minipage}[b]{\linewidth}\raggedleft
Pr(\textgreater\textbar t\textbar)
\end{minipage} \\
\midrule
\endhead
(Intercept) & 132.1 & 0.2867 & 460.8 & 0 \\
HDD & 2.195 & 0.05096 & 43.07 & 0 \\
CDD & 2.347 & 0.5722 & 4.102 & 4.113e-05 \\
Region\_Upper Hutt & -0.4796 & 0.3036 & -1.58 & 0.1141 \\
Region\_Christchurch & -0.9073 & 0.3257 & -2.786 & 0.005348 \\
Region\_Dunedin & 12.06 & 0.3835 & 31.45 & 1.705e-212 \\
Region\_Hamilton & 8.513 & 0.5298 & 16.07 & 8.999e-58 \\
Region\_Nelson & 2.711 & 0.4806 & 5.642 & 1.7e-08 \\
Region\_Rotorua & 5.015 & 0.5462 & 9.182 & 4.597e-20 \\
Region\_Clyde & 4.53 & 0.7491 & 6.048 & 1.494e-09 \\
Region\_Palmerston North & 14.11 & 0.6652 & 21.21 & 6.519e-99 \\
Region\_Stratford & 10.36 & 0.9497 & 10.91 & 1.254e-27 \\
Region\_Napier & 6.316 & 0.8473 & 7.455 & 9.311e-14 \\
Region\_Invercargill & 3.191 & 1.758 & 1.815 & 0.06949 \\
Model\_Nissan Leaf (30 kWh) & 3.401 & 0.2524 & 13.47 & 3.276e-41 \\
Model\_Nissan Leaf (24 kWh) 2011-2012 & 12.39 & 0.3246 & 38.17 &
7.229e-309 \\
Model\_Nissan Leaf (40 kWh) & 10.68 & 0.5174 & 20.63 & 1.046e-93 \\
Model\_Nissan e-NV200 (24 kWh) & 32.71 & 0.5367 & 60.95 & 0 \\
Model\_Hyundai Ioniq (EV) & -18.32 & 0.685 & -26.75 & 3.342e-155 \\
Model\_BMW i3 & -1.335 & 0.7873 & -1.695 & 0.09006 \\
Model\_Hyundai Kona (EV) & 0.6822 & 0.86 & 0.7933 & 0.4276 \\
Model\_Renault Zoe & 11.55 & 0.8507 & 13.57 & 8.383e-42 \\
Model\_Tesla Model 3 & 10.55 & 1.022 & 10.32 & 6.485e-25 \\
Model\_Nissan Leaf (62 kWh) & 25.46 & 1.752 & 14.53 & 1.295e-47 \\
Model\_Kia Niro (EV) & 11.34 & 1.193 & 9.511 & 2.075e-21 \\
Model\_Tesla Model S & 48.38 & 1.69 & 28.63 & 4.806e-177 \\
Model\_Volkswagen e-Golf & 1.208 & 1.538 & 0.7853 & 0.4323 \\
Model\_Tesla Model-X & 104.1 & 1.296 & 80.34 & 0 \\
Model\_Kia Soul & 6.276 & 1.25 & 5.022 & 5.15e-07 \\
Model\_MG ZS EV & 22.12 & 3.9 & 5.671 & 1.439e-08 \\
Model\_Renault Kangoo (van) & 56.63 & 1.537 & 36.84 & 1.301e-288 \\
Model\_Jaguar I-PACE & 73.02 & 2.951 & 24.75 & 1.949e-133 \\
Model\_Peugeot e-208 & 10.96 & 9.581 & 1.144 & 0.2525 \\
\bottomrule
\end{longtable}

\begin{longtable}[]{@{}
  >{\raggedleft\arraybackslash}p{(\columnwidth - 6\tabcolsep) * \real{0.21}}
  >{\raggedleft\arraybackslash}p{(\columnwidth - 6\tabcolsep) * \real{0.31}}
  >{\raggedleft\arraybackslash}p{(\columnwidth - 6\tabcolsep) * \real{0.12}}
  >{\raggedleft\arraybackslash}p{(\columnwidth - 6\tabcolsep) * \real{0.24}}@{}}
\caption{Fitting linear model: consumption \textasciitilde{} HDD + CDD +
Region\_ + Model\_}\tabularnewline
\toprule
\begin{minipage}[b]{\linewidth}\raggedleft
Observations
\end{minipage} & \begin{minipage}[b]{\linewidth}\raggedleft
Residual Std. Error
\end{minipage} & \begin{minipage}[b]{\linewidth}\raggedleft
\(R^2\)
\end{minipage} & \begin{minipage}[b]{\linewidth}\raggedleft
Adjusted \(R^2\)
\end{minipage} \\
\midrule
\endfirsthead
\toprule
\begin{minipage}[b]{\linewidth}\raggedleft
Observations
\end{minipage} & \begin{minipage}[b]{\linewidth}\raggedleft
Residual Std. Error
\end{minipage} & \begin{minipage}[b]{\linewidth}\raggedleft
\(R^2\)
\end{minipage} & \begin{minipage}[b]{\linewidth}\raggedleft
Adjusted \(R^2\)
\end{minipage} \\
\midrule
\endhead
22592 & 492.2 & 0.4855 & 0.4848 \\
\bottomrule
\end{longtable}

\hypertarget{how-to-read-this-table}{%
\paragraph{How to read this table:}\label{how-to-read-this-table}}

When computing the linear model as Auckland and Nissan Leaf (24 kWh)
2013-2016 are the most common region and model they are used for the
intercept. In order to get the expected consumption of a vehicle we
start with the (Intercept) Estimate. We then add to this consumption
estimate the corresponding region and model Estimate (not needed if it
is Auckland or Nissan Leaf (24 kWh) 2013-2016). Number of HDD per day is
then multiplied by the HDD Estimate from the table and added to the
consumption estimate. Similarly for CDD days number of CDD per day is
then multiplied by the CDD Estimate from the table and added to this
consumption estimate.

The HDD term suggests that as the average number of heating degree days
per days increases by 1 the average power consumption of EVs for the
month increases by 2.19Wh/km. With a p-value of \(<2\times10^{-16}\) we
are quite confident on this value.

The CDD term suggests that as the average number of cooling degree days
per days increases by 1 the average power consumption of EVs for the
month increases by 2.35Wh/km. With a p-value of \(4.11\times10^{-5}\) we
are less confident on this value. This is likely as there is much less
data in New Zealand regarding cooling degree days.

\begin{figure}
\centering
\includegraphics{final_report_files/figure-latex/consum_den-1.pdf}
\caption{distribution of linear model coeffients (effectivly
``baseline'' consumption by region for each
region)\label{fig:consum_den}}
\end{figure}

\hypertarget{predictions}{%
\subsection{Predictions}\label{predictions}}

Assumptions made with these predictions are therefore:

\begin{itemize}
\item Regional VKT data remains relatively consistent with 2019 VKT data. 2019 is chosen as in NZ there has been a significant increase in VKT in recent years excluding 2020 as there was a significant decrease due to lockdown. As lockdown is an outlier event it would be preferable to not include this in the model so 2019 is used.
\item Regional weather data from 2017 to 2021 remains consistent with future climate of NZ.
\item Flip the Fleet's fleet is representative NZs future EV fleet.
\item Actual VKT of each region remains relatively constant throughout the year.
\end{itemize}

\begin{figure}
\centering
\includegraphics{final_report_files/figure-latex/Auckland_power-1.pdf}
\caption{Auckland Region 100\% EV Case Total Power Usage per
Month\label{fig:Auckland_power}}
\end{figure}

Combining Auckland only NZTA 2019 VKT with the consumption linear model
using Flip the Fleets vehicle make up and average weather data from
2017-2021 Figure \ref{fig:Auckland_power} the estimated power usage per
month showing a clear seasonal trend from around 127.7 GWh per month in
summer to around 137.1 GWh per month in the winter.

The upper and lower estimate are 95\% confidence interval on the linear
model. As confidence intervals are unknown for the future VKT and future
weather the confidence intervals do not include this uncertainty.
(should i just remove from the plot as do not have much meaning)

\begin{figure}
\centering
\includegraphics{final_report_files/figure-latex/NZ_power-1.pdf}
\caption{NZ 100\% EV Case Total Power Usage per
Month\label{fig:NZ_power}}
\end{figure}

Similarly NZTA 2019 regional VKT combined with the consumption linear
model using Flip the Fleets vehicle make up and average regional weather
data from 2017-2021 and adding all regions power usage Figure
\ref{fig:NZ_power} a clear a clear seasonal trend from around 416 GWh
per month in summer to around 453 GWh per month in the winter.

\begin{figure}
\centering
\includegraphics{final_report_files/figure-latex/NZ_region_power_prop-1.pdf}
\caption{All NZ Regions Monthly Proportional Change in Power Usage
Relative to its Yearly Average\label{fig:NZ_region_power_prop}}
\end{figure}

Doing this same process for all regions we can also plot each regions
proportional change in power usage relative to its yearly average.
Figure \ref{fig:NZ_region_power_prop} shows all regions follow a similar
seasonal change in power consumption. Of note regions like Northland and
Auckland appear to have less of a seasonal trend compared to regions
such as Otago and the West Coast likely due to a warmer climate leading
to increased AC usage during the summer months decreasing efficiency in
the summer month.

\hypertarget{comparison-and-incorporation-of-times-model-predictions}{%
\subsubsection{Comparison and Incorporation of Times Model
Predictions}\label{comparison-and-incorporation-of-times-model-predictions}}

To compare our predictions and see how they fit in with other well
established models, ECCA Times Model \cite{times_model} is used as a
comparison. For this, VKT and expected power usage by expected passenger
vehicle EVs for selected years between 2018 to 2060 was downloaded from
EECA. Expected consumption (Wh/km) assumed by ECCA was then calculated
by dividing power usage by VKT.

\begin{figure}
\centering
\includegraphics{final_report_files/figure-latex/vehicle_consum-1.pdf}
\caption{NZ Vehicle Average Consumption
Scenarios\label{fig:vehicle_consum}}
\end{figure}

\begin{thebibliography}{9}
\bibitem{ev_range}
\textit{To what degree does temperature impact EV range?}
\\\texttt{\url{https://www.geotab.com/blog/ev-range/}}
\bibitem{ev_highway}
\textit{Why is the range of an EV less on the freeway than the city?}
\\\texttt{\url{https://evcentral.com.au/why-is-the-range-of-an-ev-less-on-the-freeway-than-the-city/}}
\bibitem{fuel_trade}
\textit{MBIE oil trade statistics}
\\\texttt{\url{https://www.mbie.govt.nz/building-and-energy/energy-and-natural-resources/energy-statistics-and-modelling/energy-statistics/oil-statistics/}}
\bibitem{HDD_est}
\textit{Bayesian estimation of a building's base temperature for the calculation of heating degree-days}
\\\texttt{\url{https://www.sciencedirect.com/science/article/abs/pii/S0378778816312907}}
\bibitem{NZTA_VKT}
\textit{NZTA VKT data website}
\\\texttt{\url{https://www.transport.govt.nz/statistics-and-insights/fleet-statistics/vehicle-kms-travelled-vkt-2/}}
\bibitem{times_model}
\textit{ECCA Times Model}
\\\texttt{\url{https://www.eeca.govt.nz/insights/data-tools/new-zealand-energy-scenarios-times-nz/}}
\end{thebibliography}

\end{document}
